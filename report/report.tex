% This is samplepaper.tex, a sample chapter demonstrating the
% LLNCS macro package for Springer Computer Science proceedings;
% Version 2.20 of 2017/10/04
%
\documentclass[runningheads]{llncs}
%
\usepackage{graphicx}
% Used for displaying a sample figure. If possible, figure files should
% be included in EPS format.
%
% If you use the hyperref package, please uncomment the following line
% to display URLs in blue roman font according to Springer's eBook style:
% \renewcommand\UrlFont{\color{blue}\rmfamily}

\begin{document}
%
\title{PNG-Fuzzing with JQF}
%
%\titlerunning{Abbreviated paper title}
% If the paper title is too long for the running head, you can set
% an abbreviated paper title here
%
\author{Paul Kalz \and Marwin Linke \and Sebastian Schatz}
%
\institute{Humboldt University of Berlin, Germany}
\maketitle              % typeset the header of the contribution
%
\begin{abstract}
The abstract should briefly summarize the contents of the paper in
150--250 words.

\keywords{First keyword  \and Second keyword \and Another keyword.}
\end{abstract}
%
%
%
\section{Background on the File Format PNG}
\subsection{Overview}
TODO: Overview: Gebt einen kurzen Überblick über das ausgewählte Datenformat (Historie, Verwendungszweck,...)
\subsubsection{History}
\subsubsection{Use Case}
\subsection{Input Specification}
TODO: Input Specification: Beschreibt im Detail die Spezifikation des Dateiformats. Wie sind Dateien dieses Formats aufgebaut? Existiert eine formale Spezifikation? Wie ist eine Beispieldatei aufgebaut?
\subsubsection{Specifications}
\subsubsection{Structure}
\subsection{Security}
TODO: Security: Beschreibt mögliche Sicherheitslücken im Zusammenhang mit dem Datenformat. Geht dabei näher auf bereits existierende Fälle ein (case study), ggf. auch im Zusammenhang mit den von euch ausgewählten Tools (Bug-Tracker).

\section{Implementation}
\subsection{Tools}
TODO: Tools: Gebt einen kurzen Überblick über die von euch verwendeten Libraries. Beschreibt wie diese Libraries verwendet werden können um Dateien eures Datenformats zu generieren bzw. zu verarbeiten.
\subsection{Generator}
TODO: Generator: Beschreibt im Detail die Implementation eures Generators. Begründet dabei Design-Entscheidungen sowie von euch verwendete Heuristiken.
\subsection{Fuzz Driver}
TODO: Fuzz Driver: Beschreibt grob, wie ihr bei der Implementation der Test-Treibers vorgegangen seid und welche Funktionalitäten der Library mit eurem Treiber getestet werden.
\subsection{Guidance}
TODO: Guidance: Beschreibt eure Änderungen an der Suchstrategie von JQF. Geht dabei insbesondere auf die Motivation/Intuition eurer Ideen ein, d.h. weshalb ihr diese Änderungen für sinnvoll haltet.


\section{Evaluation}
TODO: Beschreibt die durchgeführten Experimente und deren Ergebnisse. Wie hoch war die erreichte Coverage? Konnten Bugs/Crashes gefunden werden? Wenn ja, welche?
\subsection{Experiments}
\subsection{Results}

\section{Result Discussion}
TODO: Versucht die Ergebnisse der Experimente zu interpretieren und zu erklären (discussion). Zieht Sie Folgerungen aus den Ergebnissen (conclusion). Beschreibt die nächsten Schritte, die durchgeführt werden müssten/könnten/sollten (future work).
\subsection{Discussion}
\subsection{Conclusion}
\subsection{Future Work}

%
% ---- Bibliography ----
%
% BibTeX users should specify bibliography style 'splncs04'.
% References will then be sorted and formatted in the correct style.
%
% \bibliographystyle{splncs04}
% \bibliography{mybibliography}
%
\begin{thebibliography}{8}
\bibitem{ref_article1}
Author, F.: Article title. Journal \textbf{2}(5), 99--110 (2016)

\bibitem{ref_lncs1}
Author, F., Author, S.: Title of a proceedings paper. In: Editor,
F., Editor, S. (eds.) CONFERENCE 2016, LNCS, vol. 9999, pp. 1--13.
Springer, Heidelberg (2016). \doi{10.10007/1234567890}

\bibitem{ref_book1}
Author, F., Author, S., Author, T.: Book title. 2nd edn. Publisher,
Location (1999)

\bibitem{ref_proc1}
Author, A.-B.: Contribution title. In: 9th International Proceedings
on Proceedings, pp. 1--2. Publisher, Location (2010)

\bibitem{ref_url1}
LNCS Homepage, \url{http://www.springer.com/lncs}. Last accessed 4
Oct 2017
\end{thebibliography}
\end{document}
